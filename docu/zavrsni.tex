\documentclass[times, utf8, zavrsni]{fer}
\usepackage{booktabs}

\begin{document}

% TODO: Navedite broj rada.
\thesisnumber{000}

\title{Razvoj plug-in komponente za učitavanje izlaznih podataka Monte Carlo simulacija}


\author{Fran Lubina}

\maketitle

% Ispis stranice s napomenom o umetanju izvornika rada. Uklonite naredbu \izvornik ako želite izbaciti tu stranicu.
\izvornik

% Dodavanje zahvale ili prazne stranice. Ako ne želite dodati zahvalu, naredbu ostavite radi prazne stranice.

\tableofcontents

\chapter{Uvod}

\chapter{Testiranje}
Testiranje se vrši pomoću aplikacije Sekuli IDE, koja snima klikove i druge interakcije sa grafičkim sučeljem te na temelju njih stvara skripte pomoću kojih se automatizira testiranje



\chapter{Instalacija}

\chapter{Zaključak}
Zaključak.

\bibliography{literatura}
\bibliographystyle{fer}

\begin{sazetak}
Program MCNP (Monte Carlo N-Particle Transport) je program za simulaciju transporta različitih čestica.
Izlaz programa je rezultat Monte Carlo simulacije nad određenom korisnični definiranom geometrijom, odnosno datoteka u formatu specifičnom za MCNP.
VisIt nativno podržava preko 100 različitih formata, međutim ne i MCNP mesh tally datoteke, što motivira razvoj korisničke plug-in komponente sa tom funkcionalnošću.
Tema ovog rada je razvoj te komponente.

\kljucnerijeci{VisIt, MCNP, mesh tally}
\end{sazetak}

\engtitle{Developing a plug-in for reading output data of Monte Carlo simulations}
\begin{abstract}
MCNP (Monte Carlo N-Particle Transport) is a particle transport simulation code.
The output of the program are the results of a Monte Carlo simulation run on some arbitrary user-defined geometry, namely a file specific to MCNP.
VisIt natively supports over a 100 different file formats, but not MCNP mesh tally files, which motivates the development of a plug-in used for that purpose.
The development of that plug-in is the subject of this thesis. 

\keywords{VisIt, MCNP, mesh tally}
\end{abstract}

\end{document}
