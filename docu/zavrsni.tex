\documentclass[times, utf8, zavrsni]{fer}
\usepackage{booktabs}
\usepackage{float}
\usepackage{enumitem}
\usepackage{longtable}
\usepackage{appendix}
\usepackage[table]{xcolor}

\begin{document}

% TODO: Navedite broj rada.
\thesisnumber{000}

\title{Razvoj plug-in komponente za učitavanje izlaznih podataka Monte Carlo simulacija}


\author{Fran Lubina}

\maketitle

% Ispis stranice s napomenom o umetanju izvornika rada. Uklonite naredbu \izvornik ako želite izbaciti tu stranicu.
\izvornik

% Dodavanje zahvale ili prazne stranice. Ako ne želite dodati zahvalu, naredbu ostavite radi prazne stranice.

\tableofcontents

\chapter{Uvod}

\chapter{Testiranje}
Testiranje se vrši pomoću pyhton skripta, odnosno ugrađenog pyhton CLI-a unutar programa VisIt.
Isti skup testova izvodi se nad nizom mesh-eva.
Rubrika greške ukazuje na greške koje su se prije pojavljivale, no u međuvremenu su ispravljene (svi testovi prolaze).
	
	\begin{longtable}{|l|p{80pt}|p{100pt}|p{120pt}|}
\hline
\textbf{rbr} & \textbf{opis} & \textbf{slijed naredbi} & \textbf{greške}\\
\hline
\endhead % This repeats the header on each page
1. & Otvaranje datoteke i učitavanje podataka &
\begin{enumerate}
    \item open
    \item reopen
    \item operators
    \item slice
    \item reopen
\end{enumerate}
& neispravno učitane vrijednosti: sve vrijednosti minimalne\\
\hline
2. & Mijenjanje skale & \begin{enumerate}
    \item plotAtts
    \item pseudocolor
    \item linear/log
    \item reopen
    \item linear/log
    \item operators
    \item slice
    \item linear/log
\end{enumerate}
& nesipravne ili neučitane vrijednosti za neke predjele u mesh-u.\\
\hline
3. & Minimalna i maksimalna vrijednost & \begin{enumerate}
    \item Controls
    \item Query
    \item MinMax
    \item operators
    \item slice
    \item operators
    \item MinMax
\end{enumerate}
& --.\\
\hline
4. & Selektiranje volumena vrijednosti u određenom rasponu & \begin{enumerate}
    \item operators
    \item isovolume
    \item reopen
    \item plotAtts
    \item pseudocolor
    \item linear/log
    \item reopen
    \item linear/log
\end{enumerate}
& --.\\
\hline
5. & Sphere tool slice & \begin{enumerate}
	\item sphere tool    
    \item operators
    \item slice
    \item plotAtts
    \item log
    \item Query
    \item MinMax
\end{enumerate}
& --.\\
\hline
6. & Dodavanje i mijenjanje svojstava mesh-a. & \begin{enumerate}
	\item add mesh
	\item plotAtts/mesh
	\item mesh color: custom
	\item opaque color: custom
	\item opacity: 60%
	\item smoothing: fast
	\item show internal zones
	\item reopen
\end{enumerate}
& --.\\
\hline
7. & Ostali operatori: Slice, ThreeSlice i SphereSelection & \begin{enumerate}
	\item operators
	\item Slice
	\item ThreeSlice
	\item reopen
	\item Sphere
	\item reopen
	\item plotAtts
	\item log/linear
\end{enumerate}
& --.\\
\hline
\end{longtable}
	
\section{Izmjene u izvornom kodu}
Zbog nemogućnosti da se automatski prepoznaju, koristeći format naziva MCNP datoteka (bez ekstenzije - "*.*") potrebno je modificirati izvorni kod.

Naime kod koji uspoređuje nazive datoteka sa uzorcima koji definiraju ispravne nazive datoteka za pojedini plugin nije u mogućnosti raditi sa uzorcima koji definiraju nizove koji smiju sadržavati sve znakove osim nekog znaka.

Potrebno je regresijsko testiranje database engine-a.
Tablica sa testiranim plugin-ovima je dostupna u dodatku A.
Svaki plugin testiran je nad datotekom sa nasumičnim imenom od maksimalno 15 znakova čija kombinacija čini valjano ime datoteke na Windowsu (brojevi, slova i par posebnih znakova) te odgovarajućom ekstenzijom.


\chapter{Instalacija}
Napomene:
\begin{itemize}
	\item installer pokrenut kao administrator
	\item x64 arhitektura skupa naredbi (svi Intel ili AMD procesori)
\end{itemize}



\chapter{Zaključak}
Zaključak.

\bibliography{literatura}
\bibliographystyle{fer}

\begin{sazetak}
Program MCNP (Monte Carlo N-Particle Transport) je program za simulaciju transporta različitih čestica.
Izlaz programa je rezultat Monte Carlo simulacije nad određenom korisnični definiranom geometrijom, odnosno datoteka u formatu specifičnom za MCNP.
VisIt nativno podržava preko 100 različitih formata, međutim ne i MCNP mesh tally datoteke, što motivira razvoj korisničke plug-in komponente sa tom funkcionalnošću.
Tema ovog rada je razvoj te komponente.

\kljucnerijeci{VisIt, MCNP, mesh tally}
\end{sazetak}

\engtitle{Developing a plug-in for reading output data of Monte Carlo simulations}
\begin{abstract}
MCNP (Monte Carlo N-Particle Transport) is a particle transport simulation code.
The output of the program are the results of a Monte Carlo simulation run on some arbitrary user-defined geometry, namely a file specific to MCNP.
VisIt natively supports over a 100 different file formats, but not MCNP mesh tally files, which warrants the development of a plug-in used for that purpose.
The development of that plug-in is the subject of this thesis. 

\keywords{VisIt, MCNP, mesh tally}
\end{abstract}

\appendix
\chapter{Tablica testiranih plugin-a pri regresijskom testiranju}
\begin{longtable}{|l|p{70pt}|p{100pt}|p{100pt}| p{100pt}|}
	\hline
	\textbf{rbr} & \textbf{format} & \textbf{ekstenzije} & \textbf{sa izmjenama koda} & \textbf{bez izmjena koda}\\
	\hline
	\endhead % This repeats the header on each page
	1. & ADIOS2 & .bp, .bp.sst, .bp.ssc, md.idx, md.0 & \cellcolor{yellow} ADIOS2, Silo & \cellcolor{yellow} ADIOS2, Silo\\
	\hline
	2. & volimage & .curl, 3D.ux, 3D.uy, 3D.uz, 3D.rho, 3D.p, 3D.s, 3D.div, 3D.curl, 3D.mag, 3D.veldiv, 3D.velcurl, 3D.velmag, 3D.z & \cellcolor{green} OK & \cellcolor{green}OK\\
	\hline
	3. & paraDIS\_tecplot & .fld, .field, .cyl, .cylinder, .dat & \cellcolor{green} OK & \cellcolor{green} OK\\
	\hline
	4. & unv & .unv, .unv.gz, .iv, .msh & \cellcolor{green} OK & \cellcolor{green} OK\\
	\hline
	5. & MFEM & .mfem\_root, .mesh & \cellcolor{yellow} MFEM, Silo & \cellcolor{yellow} MFEM, Silo\\
	\hline
	6. & ffp & .rcs, .rcs.gz, .ffp, .ffp.gz & \cellcolor{green} OK & \cellcolor{green} OK\\
	\hline
	7. & ZeusMP & .rcs, .rcs.gz, .ffp, .ffp.gz & \cellcolor{yellow} ZeusMP, Silo & \cellcolor{yellow} ZeusMP, Silo\\
	\hline
	8. & COSMOS & .cosmos & \cellcolor{yellow} COSMOS, Silo & \cellcolor{yellow} COSMOS, Silo\\
	\hline
	\hline
	9. & lata & .lml, .lata & \cellcolor{yellow} lata, Silo & \cellcolor{yellow} lata, Silo\\
	\hline
	10. & XDMF & .xmf, .xdmf & \cellcolor{yellow} XDMF, Silo & \cellcolor{yellow} XDMF, Silo\\
	\hline
	11. & RAGE & .xmf, .xdmf & \cellcolor{yellow} RAGE, Silo & \cellcolor{yellow} RAGE, Silo\\
	\hline
	12. & paraDIS & .prds, .dat, .data, .meta & \cellcolor{green} OK & \cellcolor{green} OK \\
	\hline
\end{longtable}

	stupac "Sa izmjenama koda" su rezultati testova na verziji VisIt 3.3.3 sa preinakama u kodu sa database engine, a "bez izmjena" je VisIt 3.1.4 instaliran sa LLNL stranice.

\end{document}
