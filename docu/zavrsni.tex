\documentclass[times, utf8, zavrsni]{fer}
\usepackage{booktabs}
\usepackage{float}
\usepackage{enumitem}
\usepackage{longtable}

\begin{document}

% TODO: Navedite broj rada.
\thesisnumber{000}

\title{Razvoj plug-in komponente za učitavanje izlaznih podataka Monte Carlo simulacija}


\author{Fran Lubina}

\maketitle

% Ispis stranice s napomenom o umetanju izvornika rada. Uklonite naredbu \izvornik ako želite izbaciti tu stranicu.
\izvornik

% Dodavanje zahvale ili prazne stranice. Ako ne želite dodati zahvalu, naredbu ostavite radi prazne stranice.

\tableofcontents

\chapter{Uvod}

\chapter{Testiranje}
Testiranje se vrši pomoću aplikacije Sekuli IDE, koja snima klikove i druge interakcije sa grafičkim sučeljem te na temelju njih stvara skripte pomoću kojih se automatizira testiranje.
Isti skup testova izvodi se nad nizom mreža.
	
	\begin{longtable}{|l|p{80pt}|p{100pt}|p{120pt}|}
\hline
rbr & opis & slijed naredbi & moguće greške\\
\hline
\endhead % This repeats the header on each page
1. & Otvaranje datoteke i učitavanje podataka &
\begin{enumerate}
    \item open
    \item reopen
    \item operators
    \item slice
    \item reopen
\end{enumerate}
& neispravno učitane vrijednosti: sve vrijednosti minimalne\\
\hline
2. & Mijenjanje skale & \begin{enumerate}
    \item plotAtts
    \item pseudocolor
    \item linear/log
    \item reopen
    \item linear/log
    \item operators
    \item slice
    \item linear/log
\end{enumerate}
& nesipravne ili neučitane vrijednosti za neke predjele u mesh-u.\\
\hline
3. & Minimalna i maksimalna vrijednost & \begin{enumerate}
    \item Controls
    \item Query
    \item MinMax
    \item operators
    \item slice
    \item operators
    \item MinMax
\end{enumerate}
& --.\\
\hline
4. & Selektiranje volumena vrijednosti u određenom rasponu & \begin{enumerate}
    \item operators
    \item isovolume
    \item reopen
    \item plotAtts
    \item pseudocolor
    \item linear/log
    \item reopen
    \item linear/log
\end{enumerate}
& --.\\
\hline
5. & Sphere tool slice & \begin{enumerate}
	\item sphere tool    
    \item operators
    \item slice
    \item plotAtts
    \item log
    \item Query
    \item MinMax
\end{enumerate}
& --.\\
\hline
6. & Dodavanje i mijenjanje svojstava mesh-a. & \begin{enumerate}
	\item add mesh
	\item plotAtts/mesh
	\item mesh color: custom
	\item opaque color: custom
	\item opacity: 60%
	\item smoothing: fast
	\item show internal zones
	\item reopen
\end{enumerate}
& --.\\
\hline
7. & Ostali operatori: Slice, ThreeSlice i SphereSelection & \begin{enumerate}
	\item operators
	\item Slice
	\item ThreeSlice
	\item reopen
	\item Sphere
	\item reopen
	\item plotAtts
	\item log/linear
\end{enumerate}
& --.\\
\hline
\end{longtable}
	


\chapter{Instalacija}

\chapter{Zaključak}
Zaključak.

\bibliography{literatura}
\bibliographystyle{fer}

\begin{sazetak}
Program MCNP (Monte Carlo N-Particle Transport) je program za simulaciju transporta različitih čestica.
Izlaz programa je rezultat Monte Carlo simulacije nad određenom korisnični definiranom geometrijom, odnosno datoteka u formatu specifičnom za MCNP.
VisIt nativno podržava preko 100 različitih formata, međutim ne i MCNP mesh tally datoteke, što motivira razvoj korisničke plug-in komponente sa tom funkcionalnošću.
Tema ovog rada je razvoj te komponente.

\kljucnerijeci{VisIt, MCNP, mesh tally}
\end{sazetak}

\engtitle{Developing a plug-in for reading output data of Monte Carlo simulations}
\begin{abstract}
MCNP (Monte Carlo N-Particle Transport) is a particle transport simulation code.
The output of the program are the results of a Monte Carlo simulation run on some arbitrary user-defined geometry, namely a file specific to MCNP.
VisIt natively supports over a 100 different file formats, but not MCNP mesh tally files, which motivates the development of a plug-in used for that purpose.
The development of that plug-in is the subject of this thesis. 

\keywords{VisIt, MCNP, mesh tally}
\end{abstract}

\end{document}
